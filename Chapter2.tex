\newpage
%----------------------------------------------------------------------------------------------------
\section{Dealing with Cards} 
%----------------------------------------------------------------------------------------------------
\vspace{10cm}
\hrule

\lhQ What is something simple we could begin to solve?
\lhA Comparing cards.
\lhN How do we proceed?
\lhA Write a failing test.
\lhN Ok. Let compare a \clubs 6 and a \spades 6. These two cards should be considered equals in value.
\begin{lstlisting}[frame=single]
module Tests
where 
import Test.HUnit

main = runTestTT $ TestList 
       [compare "6♣" "6♠" ~?= EQ]
\end{lstlisting} % $
What is the result?
\lhA \failure The result is failure: 
\begin{small} 
\begin{alltt}
expected: EQ
 but got: LT
\end{alltt}
\end{small}
But it's not a big matter, since we're comparing \il!String!s when we should compare \il!Card!s.
\lhN What is a \il!Card!?
\lhA It's a new data type.
\lhN How do I create values of this type?
\lhA Pretend you have a function from \il!String! to \il!Card!.
\lhN Ok. I'll just call that function \il!card! :
\begin{lstlisting}[frame=single]
main = runTestTT $ TestList 
       [compare (card "6♣") (card "6♠") ~?= EQ] 
\end{lstlisting} % $
What now?
\lhA \error Error, as expected. Let me just write the function.
\begin{lstlisting}[frame=single]
module PokerHand
where

card :: String -> Card
\end{lstlisting}
\lhN \error This results in two errors: 
\begin{small}
\begin{alltt}
The type signature for `card' lacks an 
accompanying binding

Not in scope: type constructor or class `Card'
\end{alltt}
\end{small}
Can you write provide the missing parts ?
\lhA \error OK, this is the \il!Card! type:
\begin{lstlisting}[frame=single]
data Card = C 
\end{lstlisting}
It has just a single value, \il!C!. And we implement the function
\begin{lstlisting}[frame=single]
card :: String -> Card
card _ = C
\end{lstlisting}
which is just producing the single value.
\lhN \error Now we have another error:
\begin{small}
\begin{alltt}
No instance for (Ord Card) arising from a use 
of `compare' 
Possible fix: add an instance declaration 
for (Ord Card)
\end{alltt}
\end{small}
Should we make the suggested fix?
\lhA \error Sure:
\begin{lstlisting}[frame=single]
data Card = C deriving (Ord)

card :: String -> Card
card _ = C
\end{lstlisting}
\lhN \error Now we have this:
\begin{small}
\begin{alltt}
No instance for (Eq Card) arising from a use 
of `compare' 
Possible fix: add an instance declaration
for (Eq Card)
\end{alltt}
\end{small}
\hspace*{\fill} 
\lhA \error Again, let's do what the compiler suggests 
\begin{lstlisting}[frame=single]
data Card = C deriving (Ord,Eq)

card :: String -> Card
card _ = C
\end{lstlisting}
\success And the test passes.
\lhN Of course, this is just a \emph{fake} implementation of the function \il!card!.
\lhA Then write another test.
\lhN Here you go:
\begin{lstlisting}[frame=single]
main = runTestTT $ TestList 
       [compare (card "6♣") (card "6♠") ~?= EQ
       ,compare (card "6♣") (card "5♠") ~?= GT]
\end{lstlisting} % $
How do we make it pass?
\lhA \error We have to compare the rank values of the cards, so we should store this value in the \il!Card! type:
\begin{lstlisting}[frame=single]
data Card = C Value deriving (Ord,Eq)
type Value = Int

card :: String -> Card
card _ = C 0
\end{lstlisting}
\failure Of course, the test now fails, as we must calculate the real value instead of returning zero. Let's think..
\lhN \failure Just make the test pass. I don't like having to think on a red bar.
\lhA \failure Let's play ``\emph{fake it 'til you make it}'' then:
\begin{lstlisting}[frame=single]
card :: String -> Card
card ['6',_] = C 6
card ['5',_] = C 5 
\end{lstlisting}
\success Now it's obvious.
\lhN \success Indeed, just convert from \il!Char! to \il!Int!, using the \il!ord! function. Do it.
\lhA \success Ok!
\begin{lstlisting}[frame=single]
module PokerHand
where
import Char

data Card = C Value deriving (Ord,Eq)
type Value = Int

card :: String -> Card
card [c,_] = C $ (ord c) - (ord '0')
\end{lstlisting} % $
\success Done.
\lhN Done? I think I have a new test to write. But first I'll do some refactoring, too.
\begin{lstlisting}[frame=single]
main = runTestTT $ TestList 
       [compare (card "6♣") (card "6♠") ~?= EQ
       ,compare (card "6♣") (card "5♠") ~?= GT]
\end{lstlisting} % $
You know about \il!comparing! right?
\lhA Yes, and so does $GHCI$: \\

\il!comparing :: (Ord a) => (b -> a) -> b -> b -> Ordering! \\
\il! -- Defined in Data.Ord! \\


\il!comparing! takes a function from a type \il!b! to an ordered type \il!a!, two values of type \il!b! and gives the comparison using the given function.
\lhN Yes, so I can compare \il!String!s using the \il!card! function:
\begin{lstlisting}[frame=single]
import Data.Ord (comparing)

main = runTestTT $ TestList 
       [comparing card "6♣" "6♠" ~?= EQ
       ,comparing card "6♣" "5♠" ~?= GT]
\end{lstlisting} % $
\hspace*{\fill}
\lhA \success Nice!
\lhN Now for my new test:
\begin{lstlisting}[frame=single]
main = runTestTT $ TestList 
       [comparing card "6♣" "6♠" ~?= EQ
       ,comparing card "6♣" "5♠" ~?= GT
       ,comparing card "T♣" "J♠" ~?= LT]
\end{lstlisting} % $
\failure We're expecting \il!LT! but get \il!GT!. Can you make it pass?
\lhA
\failure Sure:
\begin{lstlisting}[frame=single]
card :: String -> Card
card ['J',_] = C 11
card ['T',_] = C 10
card [c,_] = C $ (ord c) - (ord '0')
\end{lstlisting} % $
\success We just have to add special cases.
\lhN Good. Here's a new one:
\begin{lstlisting}[frame=single]
main = runTestTT $ TestList 
       [comparing card "6♣" "6♠" ~?= EQ
       ,comparing card "6♣" "5♠" ~?= GT
       ,comparing card "T♣" "J♠" ~?= LT
       ,comparing card "K♣" "A♣" ~?= LT]
\end{lstlisting} % $
\hspace*{\fill} 
\lhA \failure Ok. 
\begin{lstlisting}[frame=single]
card :: String -> Card
card ['A',_] = C 14
card ['K',_] = C 13
card ['J',_] = C 11
card ['T',_] = C 10
card [c,_] = C $ (ord c) - (ord '0')
\end{lstlisting} % $
\success That's easy: give each card its value.
\lhN We forgot the Queen value:
\begin{lstlisting}[frame=single]
main = runTestTT $ TestList 
       [comparing card "6♣" "6♠" ~?= EQ
       ,comparing card "6♣" "5♠" ~?= GT
       ,comparing card "T♣" "J♠" ~?= LT
       ,comparing card "K♣" "A♣" ~?= LT
       ,comparing card "Q♣" "K♣" ~?= LT]
\end{lstlisting} % $
Can you add it?
\lhA \failure Sure:
\begin{lstlisting}[frame=single]
card :: String -> Card
card ['A',_] = C 14
card ['K',_] = C 13
card ['Q',_] = C 12
card ['J',_] = C 11
card ['T',_] = C 10
card [c,_] = C $ (ord c) - (ord '0')
\end{lstlisting} % $
\success And we are done with card values.
\lhN We are, but these tests are a bit heavy. Can you think of a way to avoid repeating all these comparisons?
\lhA Yes: we could test the sorting of a deck.
\lhN  
\begin{lstlisting}[frame=single]
module Tests
where 
import Test.HUnit
import PokerHand
import Data.List (sort)

ud = map card ["A♣","2♣","T♣","K♣","9♣","Q♣","J♣"]
sd = map card ["2♣","9♣","T♣","J♣","Q♣","K♣","A♣"]

main = runTestTT $ TestList 
       [sort ud  ~?= sd]
\end{lstlisting} % $
Is that what you mean?
\lhA \error Yes, but we have a new problem.
\lhN \error Indeed:
\begin{small}
\begin{alltt}
No instance for (Show Card) arising from a use 
of `~?='
Possible fix: add an instance declaration 
for (Show Card)
\end{alltt}
\end{small}
Should we follow the suggestion?
\lhA \error No. I don't think the \il!Card! type should derive the \il!Show! class just for testing reasons.
\lhN Then should we get back to the previous version of our tests?
\lhA I have a better idea: instead of comparing lists of \il!Card!s we can compare lists of \il!String!s.
\lhN \error Comparing the \il!String!s ? Ok:
\begin{lstlisting}[frame=single]
ud = ["A♣","2♣","T♣","K♣","9♣","Q♣","J♣"]
sd = ["2♣","9♣","T♣","J♣","Q♣","K♣","A♣"]

main = runTestTT $ TestList 
       [sort ud  ~?= sd]
\end{lstlisting} % $
\failure But now the test fails:
\begin{small}
\begin{alltt}
expected: ["2\monoclubs","9\monoclubs","T\monoclubs","J\monoclubs","Q\monoclubs","K\monoclubs","A\monoclubs"]
 but got: ["2\monoclubs","9\monoclubs","A\monoclubs","J\monoclubs","K\monoclubs","Q\monoclubs","T\monoclubs"]
\end{alltt}
\end{small}
Do you see why?
\lhA \failure Of course: we don't use \il!Card!s any more! We should compare the \il!String!s using the \il!card! function. The function \\

\il!sortBy :: (a -> a -> Ordering) -> [a] -> [a]! \\

allows us to do that.
\lhN You mean like this:
\begin{lstlisting}[frame=single]
import Data.Ord (comparing)
import Data.List (sortBy)

ud = ["A♣","2♣","T♣","K♣","9♣","Q♣","J♣"]
sd = ["2♣","9♣","T♣","J♣","Q♣","K♣","A♣"]

main = runTestTT $ TestList 
       [sortBy (comparing card) ud  ~?= sd]
\end{lstlisting} % $
\hspace*{\fill}
\lhA \success Yes!
\lhN I wonder what would the test show if it failed. Let's falsify it:
\begin{lstlisting}[frame=single]
import Data.Ord (comparing)
import Data.List (sortBy)

ud = ["3♣","2♣","T♣","K♣","9♣","Q♣","J♣"]
sd = ["2♣","9♣","T♣","J♣","Q♣","K♣","A♣"]

main = runTestTT $ TestList 
       [sortBy (comparing card) ud  ~?= sd]
\end{lstlisting} % $
I just changed the first value of the unsorted desk. 
\lhA \failure Here is what the message says:
\begin{small}
\begin{alltt}
expected: ["2\monoclubs","9\monoclubs","T\monoclubs","J\monoclubs","Q\monoclubs","K\monoclubs","A\monoclubs"]
 but got: ["2\monoclubs","3\monoclubs","9\monoclubs","T\monoclubs","J\monoclubs","Q\monoclubs","K\monoclubs"]
\end{alltt}
\end{small}
The test properly outputs the results as a list of \il!String!s. You can un-falsify the test now.
\lhN Yes.
\lhA Oh, and using \il!words! for the definition of our decks would make the code prettier.
\lhN You are right. So this is the test code:
\begin{lstlisting}[frame=single]
module Tests
where 
import Test.HUnit
import PokerHand
import Data.Ord (comparing)
import Data.List (sortBy)

ud = words "A♣ 2♣ T♣ K♣ 9♣ Q♣ J♣"
sd = words "2♣ 9♣ T♣ J♣ Q♣ K♣ A♣"

main = runTestTT $ TestList 
       [sortBy (comparing card) ud  ~?= sd]
\end{lstlisting} % $
\hspace*{\fill}
\lhA \success And this is the tested code:
\begin{lstlisting}[frame=single]
module PokerHand
where
import Char

data Card = C Value deriving (Ord,Eq)
type Value = Int

card :: String -> Card
card ['A',_] = C 14
card ['K',_] = C 13
card ['Q',_] = C 12
card ['J',_] = C 11
card ['T',_] = C 10
card [c,_] = C $ (ord c) - (ord '0')
\end{lstlisting} %$ 
\lhN Are we done with comparing \il!Cards!?
\lhA Not yet, but it's time for a break.
\lhend
