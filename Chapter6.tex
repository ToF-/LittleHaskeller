\newpage
%----------------------------------------------------------------------------------------------------
\section{Straight \& Flush} 
%----------------------------------------------------------------------------------------------------
\vspace{10cm}
\hrule

\lhQ So far, our \il!hand! function is still not correct with regard to the rules of Poker.
\lhA Agreed. At least three rankings are missing.
\lhN What are they ?
\lhA The \emph{Straight}, the \emph{Flush}, and the \emph{Straight Flush}.
\lhN What about the \emph{Royal Flush}?
\lhA It's another name for the highest \emph{Straight Flush}.
\lhN I'll begin with a test for a \emph{Straight} beating any \emph{Three of a Kind}. What is an example of the lowest possible \emph{Straight}?
\lhA \diamonds5 \clubs4 \diamonds3 \hearts2 \spades1. This is a special case, though, because the ace is not the highest value in that hand.
\lhN Then, let's begin with the general case and use \spades6 \diamonds5 \clubs4 \diamonds3 \hearts2 instead:
\begin{lstlisting}[frame=single]
        ,"6♠ 5♦ 4♣ 3♦ 2♥" `beat` "A♣ A♥ A♦ K♣ Q♠"]
\end{lstlisting}
\hspace*{\fill}
\lhA \failure We'll use the same routine as before. First, describe the new \il!Hand! value:
\begin{lstlisting}[frame=single]
data Hand = HighCard [Card]
          | Pair     [Card]
          | ThreeOfAKind [Card]
          | Straight [Card]
          | FullHouse [Card]
          | FourOfAKind [Card]
            deriving (Ord,Eq)
\end{lstlisting}
\error then completing the \il!hand! function.
\lhN Do you know how to recognize a \il!Straight!?
\lhA \error Yes: it's like a \il!HighCard!, meaning that every value is distinct, but the values are in sequence, meaning that the highest value minus the lowest should equal 4. I'll add this criteria as guard:
\lhN Go on.
\lhA
\begin{lstlisting}[frame=single]
ranking :: [[Card]] -> Hand
ranking [[a,b,c,d],[e]]       = FourOfAKind [a,b,c,d,e]
ranking [[a,b,c],[d,e]]       = FullHouse [a,b,c,d,e]
ranking [[a,b,c],[d],[e]]     = ThreeOfAKind [a,b,c,d,e]
ranking [[a,b],[c],[d],[e]]   = Pair     [a,b,c,d,e]
ranking [[a],[b],[c],[d],[e]] 
    | value a - value e == 4 = Straight [a,b,c,d,e] 
ranking [[a],[b],[c],[d],[e]] = HighCard [a,b,c,d,e] 
\end{lstlisting}
\success And now the test is passing.
\lhN Good. What about the special case where the ace is the lowest? I'll add the test:
\begin{lstlisting}[frame=single]
       ,"5♠ 4♦ 3♣ 2♦ A♥" `beat` "A♣ A♥ A♦ K♣ Q♠"]
\end{lstlisting}
\failure The test fails. Can you make it pass ?
\lhA \failure Yes, we just have to add the same pattern with a new guard for the case where the highest card is an ace and the next one is a five:
\begin{lstlisting}[frame=single]
ranking :: [[Card]] -> Hand
ranking [[a,b,c,d],[e]]       = FourOfAKind [a,b,c,d,e]
ranking [[a,b,c],[d,e]]       = FullHouse [a,b,c,d,e]
ranking [[a,b,c],[d],[e]]     = ThreeOfAKind [a,b,c,d,e]
ranking [[a,b],[c],[d],[e]]   = Pair     [a,b,c,d,e]
ranking [[a],[b],[c],[d],[e]] 
    | value a - value e == 4 = Straight [a,b,c,d,e] 
ranking [[a],[b],[c],[d],[e]] 
    | value a == 14 && value b == 5 = Straight [b,c,d,e,a] 
ranking [[a],[b],[c],[d],[e]] = HighCard [a,b,c,d,e] 
\end{lstlisting}
\success Note that I order the cards in the value differently, so that the ace is at the last position. 
\lhend

