\newpage
%----------------------------------------------------------------------------------------------------
\section{Finding Hands} 
%----------------------------------------------------------------------------------------------------
\vspace{10cm}
\hrule

\lhQ We know how to compute a hand's ranking, and print that ranking. What do we need to do now?
\lhA We need to find the five card hand with the best ranking in an arbitrary list of cards.
\lhN How do we do that?
\lhA Just write a failing test.
\lhN Ok. Here we go:
\begin{lstlisting}[frame=single]
maxRanking "6♥ 6♦ 6♠ 6♣ K♠ K♦" ~?= Nothing
\end{lstlisting}
In that case, the result is \il!Nothing! because there are less than seven cards in the string. You know about \il!Nothing!, right ?
\lhA \error Yes. 
\begin{lstlisting}[frame=single]
maxRanking :: String -> Maybe Ranking
maxRanking _ = Nothing
\end{lstlisting}
\success Your test is implying that \il!maxRanking! consumes a \il!String! and returns, \il!Maybe!, a \il!Ranking!. 
\lhN That is correct. Here's another one:
\begin{lstlisting}[frame=single]
maxRanking "6♣ 4♦ A♣ 3♠ K♠" ~?= Just HighCard
\end{lstlisting}
\lhA \failure I'll make it pass as fast as I can:
\begin{lstlisting}[frame=single]
maxRanking :: String -> Maybe Ranking
maxRanking s | length (cards s) < 5 = Nothing
maxRanking s = Just HighCard
\end{lstlisting}
\success We just ignore lists of less than 5 cards.
\lhN Ok. But there is still a \emph{fake}. Here's a new test:
\begin{lstlisting}[frame=single]
maxRanking "6♣ 6♦ A♣ 3♠ K♠" ~?= Just Pair
\end{lstlisting}
\lhA Easy: we just yield the ranking
\begin{lstlisting}[frame=single]
maxRanking :: String -> Maybe Ranking
maxRanking s | length (cards s) < 5 = Nothing
maxRanking s = Just (ranking (hand s))
\end{lstlisting}
\error Uh oh.
\lhN
\begin{small}
\begin{verbatim}
Not in scope: ranking
\end{verbatim}
\end{small}
We don't have a function \il!ranking!. We had one, but we renamed it.
\lhA Ok, here's the needed function:
\begin{lstlisting}[frame=single]
ranking :: Hand -> Ranking
ranking (H r _) = r
\end{lstlisting}
\success and now the test passes.
\lhN Ok. Here's the really complicated case:
\begin{lstlisting}[frame=single]
maxRanking "9♣ A♥ K♠ 3♣ K♦ 9♦ 6♦" ~?= Just TwoPairs
\end{lstlisting}
\lhA Here's a simple \emph{fake} solution.
\begin{lstlisting}[frame=single]
maxRanking :: String -> Maybe Ranking
maxRanking s | length (words s) < 5 = Nothing
maxRanking s | length (words s) == 5 = Just (ranking (hand s))
maxRanking _ = Just TwoPairs
\end{lstlisting}
\success Now we need to find the general implementation.
\lhN
There are several possible five card hands we can form with these seven cards. Do you know how much?
\lhA Yes, ${{7}\choose{5}} = \frac{7\times6 \times 5 \times 4 \times 3}{5 \times 4 \times 3 \times 2 \times 1} = \frac{2520}{120} = 21$
\lhN Do you know how to find them?
\lhA Yes. I can use the \il!subsequences! function. For example\\ 
\il!subsequences "CAT"! \\
gives: \\
\il!["","C","A","CA","T","CT","AT","CAT"]!
\lhN What if we want only two letters subsequences ?
\lhA I suppose applying \il!filter ((2==) . lenght)! on the list would do the trick.
\lhN Then we have to find the best hand.
\lhA Oh, that's the simplest part.
\lhN Let's begin with that part, then.
\lhA Allright. Suppose we have computed some sublists already:
\begin{lstlisting}[frame=single]
maxRanking :: String -> Maybe Ranking
maxRanking s | length (words s) < 5 = Nothing
maxRanking s | length (words s) == 5 = Just (ranking (hand s))
maxRanking _ = 
    let ls = ["K♠ 3♣ K♦ 9♦ 6♦", 
              "9♣ K♠ 3♣ K♦ 9♦", 
              "9♣ A♥ K♠ 3♣ K♦"]
    in Just TwoPairs
\end{lstlisting}
\success Here we don't have all the possible five card sublists, but these will suffice.
\lhN Go on.
\lhA Then, to find the best ranking hand from this list is easy:
\begin{lstlisting}[frame=single]
maxRanking :: String -> Maybe Ranking
maxRanking s | length (words s) < 5 = Nothing
maxRanking s | length (words s) == 5 = Just (ranking (hand s))
maxRanking _ = 
    let ls = ["K♠ 3♣ K♦ 9♦ 6♦", 
              "9♣ K♠ 3♣ K♦ 9♦", 
              "9♣ A♥ K♠ 3♣ K♦"]
        max  = maximum . map ranking . map hand
    in Just (max ls)
\end{lstlisting}
\success And we're done for that part.
\lhN Ok. Now for finding the sublists?
\lhA Hmm. First we need a helper function to create 5 item sublists from a list
\lhN Ok. What about a function such as:
\begin{lstlisting}[frame=single]
subLists 2 "CAT" ~?= ["CA","CT","AT"]
\end{lstlisting}
I introduce a variable, because I don't really want to write the case for 21 sublists.
\lhA \error Nice idea. Here's the function:
\begin{lstlisting}[frame=single]
subLists :: Int -> [a] -> [[a]]
subLists n = filter ((n ==) . length) . subsequences 
\end{lstlisting}
\success It's a bit more general than needed, thought.
\lhN Ok. Now use the function to find hands.
\lhA Not too fast. We cannot apply \il!subLists 5! on our input: \il!"9♣ A♥ K♠ 3♣ K♦ 9♦ 6♦"!.
\lhN How can we use it then?
\lhA We can use it on a list of cards made with this input. Thus a new helper function:
\begin{lstlisting}[frame=single]
fiveCardLists :: String -> [[Card]]
fiveCardLists = (subLists 5) . cards
\end{lstlisting}
\lhN Ok. Now we can use this function. Right?
\lhA No: our hand function expects a \il!String!:
\begin{lstlisting}[frame=single]
hand :: String -> Hand
hand =     cards
       >>. rSortBy (comparing value)
       >>. groupBy (same value)
       >>. rSortBy (comparing length)
       >>. rank
       >>. promoteStraight
       >>. promoteFlush    
\end{lstlisting}
\lhN Then make it expect a list of \il!Card!s.
\lhA Ok:
\begin{lstlisting}[frame=single]
hand :: [Card] -> Hand
hand = rSortBy (comparing value)
       >>. groupBy (same value)
       >>. rSortBy (comparing length)
       >>. rank
       >>. promoteStraight
       >>. promoteFlush    
\end{lstlisting}
\error And now we have errors.
\lhN Yes:
\begin{small}
\begin{verbatim}
Couldn't match expected type `Card' 
  against inferred type `Char'
      Expected type: [Card]
      Inferred type: String
\end{verbatim}
\end{small}
\lhA \error I see. We need to change the calls to \il!hand! here:
\begin{lstlisting}[frame=single]
maxRanking :: String -> Maybe Ranking
maxRanking s | length (words s) < 5 = Nothing
maxRanking s | length (words s) == 5 = Just (ranking (hand s))
maxRanking _ = 
    let hs = ["K♠ 3♣ K♦ 9♦ 6♦", 
              "9♣ K♠ 3♣ K♦ 9♦", 
              "9♣ A♥ K♠ 3♣ K♦"]
        max  = maximum . map ranking . map hand
    in Just (max hs)
\end{lstlisting}
\lhN Exact. We need to call our \il!cards! function.
\lhA \error Let's do it.
\begin{lstlisting}[frame=single]
maxRanking :: String -> Maybe Ranking
maxRanking s | length (words s) < 5 = Nothing
maxRanking s | length (words s) == 5 = Just (ranking (hand (cards s)))
maxRanking _ = 
    let ls = ["K♠ 3♣ K♦ 9♦ 6♦", 
              "9♣ K♠ 3♣ K♦ 9♦", 
              "9♣ A♥ K♠ 3♣ K♦"]
        max  = maximum . map ranking . map (hand . cards) 
    in Just (max ls)
\end{lstlisting}
\error We still have an error, though.
\lhN It's in the tests. I need to change this:
\begin{lstlisting}[frame=single]
beat h g = comparing hand  h g ~?= GT
\end{lstlisting}
into this:
\begin{lstlisting}[frame=single]
beat h g = comparing (hand . cards) h g ~?= GT
\end{lstlisting}
\success And everything is back to normal.
\lhA Good.
\lhN Now can we use our clever \il!fiveCardLists! function?
\lhA I think so:
\begin{lstlisting}[frame=single]
maxRanking :: String -> Maybe Ranking
maxRanking s | length (words s) < 5 = Nothing
maxRanking s | length (words s) == 5 = Just (ranking (hand (cards s)))
maxRanking s = 
    let hs = fiveCardLists s
        max  = maximum . map ranking . map hand 
    in Just (max hs)
\end{lstlisting}
\success Yes!
\lhN Ok. Now is a good time to clean the code, don't you think?
\lhA You're right. First we can get rid of the intermediate pattern:
\begin{lstlisting}[frame=single]
maxRanking :: String -> Maybe Ranking
maxRanking s | length (words s) < 5 = Nothing
maxRanking s = 
    let hs = fiveCardLists s
        max  = maximum . map ranking . map hand 
    in Just (max hs)
\end{lstlisting}
\success because the general strategy works for this case as well.
\lhN Good. Go on.
\lhA Next, I will expose some similarities:
\begin{lstlisting}[frame=single]
maxRanking :: String -> Maybe Ranking
maxRanking s | fiveCardLists s == [] = Nothing
maxRanking s = 
    let hs = fiveCardLists s
        max  = maximum . map ranking . map hand 
    in Just (max hs)
\end{lstlisting}
\success so that we don't deal at the \il!String! level in this function.
\lhN I see where you are going. 
\lhA Then, we can simplify the whole function:
\begin{lstlisting}[frame=single]
maxRanking :: String -> Maybe Ranking
maxRanking s = case (subLists 5 (cards s)) of
                  [] -> Nothing
                  hs -> Just (max hs)
    where max = maximum . map ranking . map hand
\end{lstlisting}
\success and we get rid of the \il!fiveCardLists! helper. 

\lhN Nice. Here are the test:
\begin{lstlisting}[frame=single]
module Tests
where 
import Test.HUnit
import PokerHand
import Data.Ord (comparing)
import Data.List (sort,sortBy)

ud = words "A♣ 2♣ T♣ K♣ 9♣ Q♣ J♣"
sd = words "2♣ 9♣ T♣ J♣ Q♣ K♣ A♣"

main = runTestTT $ TestList 
       [sortBy (comparing card) ud  ~?= sd
       ,map suit (cards "A♣ A♦ A♥ A♠") ~?= "♣♦♥♠"
       ,flush (cards "A♣ T♣ 3♣ 4♣ 2♣") ~?= True
       ,flush (cards "A♠ T♣ 3♣ 4♣ 2♣") ~?= False
       ,flush (cards "A♠ T♠ 3♠ 4♠ 2♠") ~?= True
       ,"6♣ 4♦ A♣ 3♠ K♠" `beat` "8♥ J♥ 7♦ 5♥ 6♣"
       ,"5♥ 2♦ 3♥ 4♦ 2♥" `beat` "A♥ K♥ Q♦ J♦ 9♥"
       ,"5♥ 4♦ 3♥ 2♦ 3♣" `beat` "A♥ K♥ Q♦ J♦ 9♥"
       ,"5♥ 4♦ 3♥ 3♣ 2♥" `beat` "7♦ 5♥ 3♦ 2♠ 2♦"
       ,"2♦ 2♣ 3♣ 3♠ 4♥" `beat` "A♥ A♠ K♣ Q♦ J♠"  
       ,"2♦ 2♣ 2♠ 3♥ 4♦" `beat` "A♥ A♠ K♣ K♦ J♠"
       ,"2♦ 2♠ 2♥ 2♣ 3♦" `beat` "A♥ A♦ A♠ K♥ K♠"
       ,"6♠ 5♦ 4♣ 3♦ 2♥" `beat` "A♣ A♥ A♦ K♣ Q♠"
       ,"5♠ 4♦ 3♣ 2♦ A♥" `beat` "A♣ A♥ A♦ K♣ Q♠"
       ,"6♥ 4♥ 3♥ 2♥ A♥" `beat` "A♠ K♣ Q♥ J♠ T♦"
       ,"5♥ 4♥ 3♥ 2♥ A♥" `beat` "A♦ A♠ A♥ A♠ K♥"
       ,"6♥ 5♥ 4♥ 3♥ 2♥" `beat` "A♦ A♠ A♥ A♠ K♥"
       ,show HighCard      ~?= "High Card"
       ,show Pair          ~?= "Pair"
       ,show TwoPairs      ~?= "Two Pairs"
       ,show ThreeOfAKind  ~?= "Three of a Kind"
       ,show Straight      ~?= "Straight"
       ,show Flush         ~?= "Flush"
       ,show FullHouse     ~?= "Full House"
       ,show FourOfAKind   ~?= "Four of a Kind"
       ,show StraightFlush ~?= "Straight Flush"
       ,maxRanking "6♥ 6♦ 6♠ 6♣"    ~?= Nothing
       ,maxRanking "6♣ 4♦ A♣ 3♠ K♠" ~?= Just HighCard
       ,maxRanking "6♣ 6♦ A♣ 3♠ K♠" ~?= Just Pair
       ,maxRanking "9♣ A♥ K♠ 3♣ K♦ 9♦ 6♦" ~?= 
                   Just TwoPairs]
    where
      beat h g = comparing (hand . cards) h g ~?= GT
\end{lstlisting} %$
As you can see, I refactored the code for layout. And removed the test about \il!subLists!.
\lhA Then I can make this function a private helper of the function to \il!maxRanking!.
Here is the tested code:
\begin{lstlisting}[frame=single]
module PokerHand
where
import Char
import Data.Ord
import Data.List

data Card = C { value :: Value, suit :: Suit } 
            deriving (Ord,Eq)
type Value = Int
type Suit = Char

data Hand = H Ranking [Card]
            deriving (Ord,Eq)

data Ranking = HighCard
             | Pair
             | TwoPairs
             | ThreeOfAKind
             | Straight
             | Flush
             | FullHouse
             | FourOfAKind
             | StraightFlush
            deriving (Ord,Eq)

instance (Show) Ranking
    where
      show HighCard      = "High Card"
      show Pair          = "Pair" 
      show TwoPairs      = "Two Pairs" 
      show ThreeOfAKind  = "Three of a Kind" 
      show Straight      = "Straight" 
      show Flush         = "Flush" 
      show FullHouse     = "Full House" 
      show FourOfAKind   = "Four of a Kind"
      show StraightFlush = "Straight Flush"


ranking :: Hand -> Ranking
ranking (H r _) = r
\end{lstlisting}
\dots
\lhN
\lhA
\begin{lstlisting}[frame=single]
maxRanking :: String -> Maybe Ranking
maxRanking s = case (subs (cards s)) of
                  [] -> Nothing
                  hs -> Just (max hs)
    where 
      max = maximum . map ranking . map hand
      subs = filter ((5 ==) . length) . subsequences 


card :: String -> Card
card [v,s] = C (toValue v) s
    where 
      toValue 'A' = 14
      toValue 'K' = 13
      toValue 'Q' = 12
      toValue 'J' = 11
      toValue 'T' = 10
      toValue  c  = ((ord c) - (ord '0'))

same :: (Eq a) => (t -> a) -> t -> t -> Bool
same f a b = f a == f b

flush :: [Card] -> Bool
flush (c:cs) = all (same suit c) cs


rSortBy :: (Ord a) => (a -> a -> Ordering) -> [a] -> [a]
rSortBy f = sortBy (flip f)

(>>.) :: (a -> b) -> (b -> c) -> (a -> c)
(>>.) = flip (.)

hand :: [Card] -> Hand
hand = rSortBy (comparing value)
       >>. groupBy (same value)
       >>. rSortBy (comparing length)
       >>. rank
       >>. promoteStraight
       >>. promoteFlush    

rank :: [[Card]] -> Hand
rank gs = H (calcRank gs) (concat gs)  
    where calcRank [[_,_,_,_],_]   = FourOfAKind 
          calcRank [[_,_,_],_]     = FullHouse
          calcRank [[_,_,_],_,_]   = ThreeOfAKind
          calcRank [[_,_],[_,_],_] = TwoPairs
          calcRank [[_,_],_,_,_]   = Pair    
          calcRank [_,_,_,_,_]     = HighCard 

cards :: String -> [Card]
cards = map card . words 

promoteStraight :: Hand -> Hand
promoteStraight (H Straight [a,b,c,d,e]) 
    | value a - value e == 4 = 
        H Straight [a,b,c,d,e]
promoteStraight (H HighCard [a,b,c,d,e]) 
    | value a == 14 && value b == 5 = 
        H Straight [b,c,d,e,a]
promoteStraight h = h


promoteFlush :: Hand -> Hand
promoteFlush (H HighCard cs) 
    | flush cs = H Flush cs
promoteFlush (H Straight cs) 
    | flush cs = H StraightFlush cs
promoteFlush h = h
\end{lstlisting}
\lhend

