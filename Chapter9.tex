\newpage
%----------------------------------------------------------------------------------------------------
\section{Solving the problem} 
%----------------------------------------------------------------------------------------------------
\vspace{10cm}
\hrule

\lhQ Did we solve our problem?
\lhA Not yet. 
\lhN What do we need to do, then?
\lhA Mark the lines from the input with the ranking of the hand, and suffix the best one with \il!"(winner")!.
\lhN What test should I write?
\lhA Write the simplest test you can think of.
\lhN What is the trivial case for a function that should mark lines?
\lhA No hand at all.
\lhN Ok
\begin{lstlisting}[frame=single]
markResults [Nothing] ~?= [""]
\end{lstlisting}
If we don't have a hand, then there is no mark.
\lhA \error I see. Here's the function:
\begin{lstlisting}[frame=single]
markResults :: [Maybe Ranking] -> [String]
markResults _ = [""]
\end{lstlisting}
\success Done.
\lhN Here's my next case. 
\begin{lstlisting}[frame=single]
markResults [Nothing, Just Pair] 
               ~?= ["","Pair (winner)"]
\end{lstlisting}
\lhA \failure Ok. I'll just add a pattern:
\begin{lstlisting}[frame=single]
markResults :: [Maybe Ranking] -> [String]
markResults [Nothing]   = [""]
markResults [Nothing, Just Pair] = ["", "Pair (winner)"]
\end{lstlisting}
\success It's a \emph{fake}, as usual.
\lhN Do you see a possible refactoring here?
\lhA I see a \il!map!:
\begin{lstlisting}[frame=single]
markResults :: [Maybe Ranking] -> [String]
markResults = map mark
    where mark Nothing = ""
          mark (Just Pair) = "Pair (winner)"
\end{lstlisting}
\success Refactoring done.
\lhN Here's a new case:
\begin{lstlisting}[frame=single]
markResults [Nothing, Just Pair, Just HighCard] ~?= 
                ["","Pair (winner)","High Card"]
\end{lstlisting}
\error We can have several hands. The best one is the winner.
There's \emph{non-exaustive patterns} error in our code, now.
\lhA \error Sure. Here's a fix:
\begin{lstlisting}[frame=single]
markResults :: [Maybe Ranking] -> [String]
markResults = map mark
    where mark Nothing = ""
          mark (Just Pair) = "Pair (winner)"
          mark (Just r) = show r
\end{lstlisting}
\success It's still a \emph{fake}.
\lhN How can we remove the \emph{fake}?
\lhA By comparing each value in the list with the maximum value in the list.
\begin{lstlisting}[frame=single]
markResults :: [Maybe Ranking] -> [String]
markResults rs = map mark rs
    where mark Nothing = ""
          mark v@(Just r) 
              | v == maximum rs = show r ++ " (winner)" 
          mark (Just r) = show r 
\end{lstlisting}
\success It works!
\lhN Can you remove duplication?
\lhA Yes.
\begin{lstlisting}[frame=single]
markResults :: [Maybe Ranking] -> [String]
markResults rs = map mark rs
    where mark Nothing = ""
          mark v@(Just r) = (show r) ++ if (v == maximum rs) then  " (winner)" else ""
\end{lstlisting}
\success Done.
\lhN Could we have pattern in lieu of the \il!if then else!?
\lhA Yes.
\begin{lstlisting}[frame=single]
markResults :: [Maybe Ranking] -> [String]
markResults rs = map mark rs
    where mark Nothing = ""
          mark (Just r) = (show r) ++ winner (Just r)
          winner v | v == maximum rs = " (winner)"
          winner _ = ""
\end{lstlisting}
\success Done.
\lhN And we could avoid computing the \il!maximum! at each line.
\lhA You are right.
\begin{lstlisting}[frame=single]
markResults :: [Maybe Ranking] -> [String]
markResults rs = map mark rs
    where mark Nothing = ""
          mark (Just r) = (show r) ++ winner (Just r)
          winner v | v == m = " (winner)"
          winner _ = ""
          m = maximum rs
\end{lstlisting}
\success And now we are done with marking results.
\lhN What else is missing?
\lhA Our program will have to reproduce and complete the input lines.
\lhN Ok. Here a test:
\begin{lstlisting}[frame=single]
scores ["6♥ 6♦ 6♠ 6♣",
        "6♣ 4♦ A♣ 3♠ K♠ 5♦ T♠",
        "6♣ 6♦ A♣ 3♠ K♠",
        "9♣ A♥ K♠ 3♣ K♦ 9♦ 6♦"] ~?= 
           ["6♥ 6♦ 6♠ 6♣",
            "6♣ 4♦ A♣ 3♠ K♠ 5♦ T♠ High Card",
            "6♣ 6♦ A♣ 3♠ K♠" ,
            "9♣ A♥ K♠ 3♣ K♦ 9♦ 6♦ Two Pairs (winner)"]
\end{lstlisting}
\lhA Wow. This is a big test!
\lhN And an important one, for that matter. Can we make it pass?
\lhA Let's try. First We have to find the max ranking for each hand:
\begin{lstlisting}[frame=single]
scores :: [String] -> [String]
scores input = let rs = map maxRanking input
\end{lstlisting}
Then we have to compute the marks:
\begin{lstlisting}[frame=single]
                   ms = markResults rs
\end{lstlisting}
Then we join them with a concatenation operation:
\begin{lstlisting}[frame=single]
               in zipWith (++) input ms
\end{lstlisting}
\failure Does it work?
\lhN No. The resulting lines lack a space between the input and the marks:
We expect:\\ \il!"6♣ 4♦ A♣ 3♠ K♠ 5♦ T♠__High Card"! \\ 
and we get \il!"6♣ 4♦ A♣ 3♠ K♠ 5♦ T♠High Card"!.
\lhA \failure Then \il!(++)! is not the good operation to zip the lists with. Let's write our own function:
\begin{lstlisting}[frame=single]
scores :: [String] -> [String]
scores input = let rs = map (maxRanking . cards) input
                   ms = markResults rs
               in zipWith join input ms
                   where join a b  = a ++ ' ':b
\end{lstlisting}
\failure Does it work now?
\lhN No. We have a supplementary space on the first line:
We expect \il!"6♣ 6♦ A♣ 3♠ K♠"! \\ and we get \il!"6♣ 6♦ A♣ 3♠ K♠__"!.
\lhA \failure Sure: when there is no mark, we shouldn't add that space. Let's add a pattern.
\begin{lstlisting}[frame=single]
scores :: [String] -> [String]
scores input = let rs = map (maxRanking . cards) input
                   ms = markResults rs
               in zipWith join input ms
                   where join a "" = a
                         join a b  = a ++ ' ':b
\end{lstlisting}
\success And we're done!
\lhN Are we? Here's the final test case. 
\begin{alltt}
\clubs{K} \spades{9} \spades{K} \diamonds{K} \diamonds{9} \clubs{3} \diamonds{6}
\clubs{9} \hearts{A} \spades{K} \diamonds{K} \diamonds{9} \clubs{3} \diamonds{6}
\clubs{A} \clubs{Q} \spades{K} \diamonds{K} \diamonds{9} \clubs{3}
\hearts{9} \spades{5}
\diamonds{4} \diamonds{2} \spades{K} \diamonds{K} \diamonds{9} \clubs{3} \diamonds{6}
\spades{7} \spades{T} \spades{K} \diamonds{K} \diamonds{9}
\end{alltt}
I put it in a file named \emph{game.txt}.
 
\lhA Ok. We just have to create a main program which would process this file and compute the scores.

\begin{lstlisting}[frame=single] 
module Main
where
import PokerHand
\end{lstlisting}
Let's call this program \emph{Scores.hs}.
\lhN Ok. How does the program work?
\lhA Very simple. First we get the text from the input. We have to separate this text into lines, calculate the scores, assemble the result back into a text, which we display on the output: 
\begin{lstlisting}[frame=single]
main = getContents 
       >>= lines
       >>. scores
       >>. unlines
       >>. putStrLn
\end{lstlisting}
\lhN How do we try it?
\lhA just type:
\begin{alltt}
runghc Scores <game.txt
\end{alltt}
\lhN Here's the output:
\begin{alltt}
\clubs{K} \spades{9} \spades{K} \diamonds{K} \diamonds{9} \clubs{3} \diamonds{6} Full House (winner)
\clubs{9} \hearts{A} \spades{K} \diamonds{K} \diamonds{9} \clubs{3} \diamonds{6} Two Pair
\clubs{A} \clubs{Q} \spades{K} \diamonds{K} \diamonds{9} \clubs{3} 
\hearts{9} \spades{5} 
\diamonds{4} \diamonds{2} \spades{K} \diamonds{K} \diamonds{9} \clubs{3} \diamonds{6} Flush
\spades{7} \spades{T} \spades{K} \diamonds{K} \diamonds{9} 
\end{alltt}
So, I guess it works.
\lhA \success \success \success It works! Hurray!
\lhN Let's ship our program to the customer
\lhA And have a fantastic dinner!
\lhend
\lstinputlisting[frame=single,caption=Tests.hs]{tests.hs}
\lstinputlisting[frame=single,caption=Tests.hs]{PokerHand.hs}
