\newpage
%----------------------------------------------------------------------------------------------------
\section{Printing} 
%----------------------------------------------------------------------------------------------------
\vspace{10cm}
\hrule

\lhQ What should we work on, now ?
\lhA Let's do something that is easy, for a change.
\lhN What about printing the rankings ?
\lhA That will be short and sweet.
\lhN Do you remember what the program is expected to print?
\lhA Not much.
\lhN Here's an example:
\begin{alltt}
\clubs{K} \spades{9} \spades{K} \diamonds{K} \diamonds{9} \clubs{3} \diamonds{6} Full House (winner)
\clubs{9} \hearts{A} \spades{K} \diamonds{K} \diamonds{9} \clubs{3} \diamonds{6} Two Pair
\clubs{A} \clubs{Q} \spades{K} \diamonds{K} \diamonds{9} \clubs{3} 
\hearts{9} \spades{5} 
\diamonds{4} \diamonds{2} \spades{K} \diamonds{K} \diamonds{9} \clubs{3} \diamonds{6} Flush
\spades{7} \spades{T} \spades{K} \diamonds{K} \diamonds{9} 
\end{alltt}
\lhA I see. We need to print:
\begin{itemize}
\item the line of cards we have in input
\item the ranking of the hand found in the line (except for \emph{High Cards})
\item the mention \il!"(winner)"! along with the best ranking
\end{itemize}
\hspace*{\fill}\vspace*{\fill}
\lhN Let's take care of your second item: showing the ranking.
\lhA Ok.
\lhN  Here's a test:
\begin{lstlisting}[frame=single]
showRanking (hand "6♣ 4♦ A♣ 3♠ K♠") ~?= ""
\end{lstlisting}
\lhA Easy:
\begin{lstlisting}[frame=single]
showRanking :: Hand -> String
showRanking _ = ""
\end{lstlisting}
\success And the test passes.
\lhN Here's another test, then:
\begin{lstlisting}[frame=single]
showRanking (hand "5♥ 2♦ 3♥ 4♦ 2♥") ~?= "Pair"
\end{lstlisting}
\lhA 
\begin{lstlisting}[frame=single]
showRanking :: Hand -> String
showRanking (Pair _) = "Pair"
showRanking _ = ""
\end{lstlisting}
\success Done. That's easy.
\lhN Yes, easy, and tedious. Could we skip the testing part on that feature?
\lhA Not if we abide by the rule \#1 of TTD.
\lhN Which is?
\lhA You are not allowed to write any production code unless it is to make a failing unit test pass.
\lhN But I don't want to create all these hands just so that we can test the label given to the ranking.
\lhA Then just test the label given to the ranking.
\lhN You mean I should write my tests like this:
\begin{lstlisting}[frame=single]
  ,showRanking HighCard ~?= ""
  ,showRanking Pair ~?= "Pair"
\end{lstlisting}
It doesn't sound right, though. Look at the message:
\begin{small}
\begin{verbatim}
Couldn't match expected type `Hand'
against inferred type `[Card] -> Hand'
\end{verbatim}
\end{small}
\lhA \error No, that's not right. You can't use these data constructor without a list of \il!Card!s.
But you can use them with an empty list. 
\lhN Let's try:
\begin{lstlisting}[frame=single]
       ,showRanking (HighCard [__]) ~?= ""
       ,showRanking (Pair [__]) ~?= "Pair"
\end{lstlisting}
\lhA \success Yes, that's better.
\lhN In that case, I'd rather create a single test for all ranking labels:
\begin{lstlisting}[frame=single]
       ,map showRanking [HighCard [__],
                         Pair [__],
                         TwoPairs [__],
                         ThreeOfAKind [__],
                         Straight [__],
                         Flush [__],
                         FullHouse [__],
                         FourOfAKind [__],
                         StraightFlush [__]] ~?=
        ["","Pair","Two Pairs","Three of a Kind",
         "Straight","Flush","Full House",
         "Four of a Kind","Straight Flush"]
\end{lstlisting}
\lhA Ok. Here the function \il!showRanking!:
\begin{lstlisting}[frame=single]
showRanking :: Hand -> String
showRanking (Pair _)          = "Pair" 
showRanking (TwoPairs _)      = "Two Pairs" 
showRanking (ThreeOfAKind _)  = "Three of a Kind" 
showRanking (Straight _)      = "Straight" 
showRanking (Flush _)         = "Flush" 
showRanking (FullHouse _)     = "Full House" 
showRanking (FourOfAKind _)   = "Four of a Kind"
showRanking (StraightFlush _) = "Straight Flush"
showRanking _ = ""
\end{lstlisting}
\success And your big test is passing. But this code is not quite satisfying.

\lhend
